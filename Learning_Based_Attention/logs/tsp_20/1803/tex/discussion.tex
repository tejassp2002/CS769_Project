\section{Discussion}
\label{sec:discussion}
In this work we have introduced a model and training method which both contribute to significantly improved results on learned heuristics for TSP and additionally learned strong (single construction) heuristics for multiple routing problems, which are traditionally solved by problem-specific approaches. We believe that our method is a powerful starting point for learning heuristics for other combinatorial optimization problems defined on graphs, if their solutions can be described as sequential decisions. In practice, operational constraints often lead to many variants of problems for which no good (human-designed) heuristics are available such that the ability to learn heuristics could be of great practical value.

Compared to previous works, by using attention instead of recurrence (LSTMs) we introduce invariance to the input order of the nodes, increasing learning efficiency. Also this enables parallelization, for increased computational efficiency. The multi-head attention mechanism can be seen as a message passing algorithm that allows nodes to communicate relevant information over different channels, such that the node embeddings from the encoder can learn to include valuable information about the node \emph{in the context of the graph}. This information is important in our setting where decisions relate directly to the nodes in a graph. Being a graph based method, our model has increased scaling potential (compared to LSTMs) as it can be applied on a sparse graph and operate locally.

Scaling to larger problem instances is an important direction for future research, where we think we have made an important first step by using a graph based method, which can be sparsified for improved computational efficiency. Another challenge is that many problems of practical importance have feasibility constraints that cannot be satisfied by a simple masking procedure, and we think it is promising to investigate if these problems can be addressed by a combination of heuristic learning and backtracking. This would unleash the potential of our method, already highly competitive to the popular Google OR Tools project, to an even larger class of difficult practical problems.